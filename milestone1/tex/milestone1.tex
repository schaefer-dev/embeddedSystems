% exercise sheet with header on every page for math or close subjects
\documentclass[12pt]{article}
\usepackage[utf8]{inputenc}
\usepackage{latexsym}
\usepackage{multicol}
\usepackage{fancyhdr}
\usepackage{amsfonts}
\usepackage{amsmath}
\usepackage{amssymb}
\usepackage{enumerate}
\usepackage{listings}
\usepackage{graphicx}

% Shortcuts for bb, frak and cal letters
\newcommand{\E}{\mathbb{E}}
\newcommand{\V}{\mathbb{V}}
\renewcommand{\P}{\mathbb{P}}
\newcommand{\N}{\mathbb{N}}
\newcommand{\R}{\mathbb{R}}
\newcommand{\C}{\mathbb{C}}
\newcommand{\Z}{\mathbb{Z}}
\newcommand{\Pfrak}{\mathfrak{P}}
\newcommand{\Pfrac}{\mathfrak{P}}
\newcommand{\Bfrac}{\mathfrak{P}}
\newcommand{\Bfrak}{\mathfrak{B}}
\newcommand{\Fcal}{\mathcal{F}}
\newcommand{\Ycal}{\mathcal{Y}}
\newcommand{\Bcal}{\mathcal{B}}
\newcommand{\Acal}{\mathcal{A}}

% formating
\topmargin -3.5cm
\textheight 22cm
\textwidth 16.0 cm
\oddsidemargin -0.1cm

% Fancy Header on every Page
\pagestyle{fancy}
\lhead{\textbf{Embedded Systems Milestone 1}}
\rhead{Daniel Schäfer (2549458)\\ Rafael Dewes (2548365)\\ Kevin M\"uller (2550062)}
\renewcommand{\headrulewidth}{1.2pt}
\setlength{\headheight}{110pt}

\begin{document}
\pagenumbering{gobble}
\lstset{language=C++}

\section*{Pins - Scout}

\textbf{TODO}: It might be possible that the ADC is not connected over SPI! In this case the function of I/O Clock and System Clock are switched!\\

\paragraph{Must not be driven in Combination:}
The only Pins that can not be driven simultaniously are the PINS PD4 and PC5. These PINS are used for the SPI Protocol to select either the ADC and RF module as the "receiver" of a message and only one of the chips can be selected using either PIN PD4 OR Pin PC5.\\

\small
\begin{tabular}{| c || p{30mm} | p{30mm} | p{60mm} |}
  \hline
  \textbf{PIN} & FUNCTION & alternate Functions & NOTES\\
  \hline
  \hline
  \textbf{PB4} & I/O Clock of ADC & SCK of RF module & Synchronisation Clock of Serial Peripheral Interface (Master Slave Connection/SPI).\\
  \hline
  \textbf{PB5} & MOSI lane to ADC & MOSI lane to RF & Master Output, Slave Input lane. The Slave Select lane (PD4/PC5) determines with which of the slaves (either RF module or ADC) the master communicates. This lane is used to communicate to the ADC of which photoresistor data should be read and to communicate information to the RF module.\\
  \hline
  \textbf{PB0} & MISO lane to ADC & MISO lane to RF & Master Input, Slave Output lane. The master receives either the output of the data out lane of the ADC or the output of the MISO lane of the RF.\\
  \hline
  \textbf{PD4} & RF module Chip Select & - & The Slave Select Lane PD4 detimines that the communication over the BUS is with the RF module (and not with the ADC).\\
  \hline
  \textbf{PD7} & RF module Chip Enable & - & Chip Enable Activates RX (used as receiver) or TX Mode (used as transmitter)\\
  \hline
  \textbf{PD2} & RF IRQ & - & Maskable interrupt pin. Active low. Confirms either packet arrival when package loaded into RX FIFO (RX mode) or arrival of ACK (acknowledge) packet (TX mode).\\
  \hline
  \textbf{PB1} & ADC system clock & - & Prescaled Clock source, which allows time for a conversion to happen inside the ADC (I/O Clock must remain low for at least 36 system clock cycles to allow conversion to be completed)\\
  \hline
  \textbf{PC5} & ADC Chip Select & - & The Slave Select Lane PC5 detimines that the communication over the BUS is with the ADC (and not with the RF Module).\\
  \hline
  \textbf{PD0} & Dev Port RX & - & Communication Input of UART Serial Communication (Output atmega328 - Input Dev Port)\\
  \hline
  \textbf{PD1} & Dev Port TX & - & Communication Output of UART Serial Communication (Output Dev Port - Input atmega328)\\
  \hline
\end{tabular}
\normalsize

\vspace{1cm}

\newpage
\section*{Pins - Collector}
\small
\begin{tabular}{| c || p{30mm} | p{30mm} | p{60mm} |}
  \hline
  \textbf{PIN} & FUNCTION & alternate Functions & NOTES\\
  \hline
  \hline
  \textbf{PB0} & RF SCK &  & Serial Clock given by master to synchronize communication over SPI protocol. \\
  \hline
  \textbf{PD5} & RF MOSI &  & Master Output - Slave Input lane used for SPI. This lane will e.g. be used to send package contents to the RF chip.\\
  \hline
  \textbf{PD7} & RF MISO &  & Master Input - Slave Output lane used for SPI. This lane will e.g. be used to receive package contents from the RF chip.\\
  \hline
  \textbf{PB3} & RF chip select &  & This Slave Select Lane detimines that the communication over the BUS is with the RF module\\
  \hline
  \textbf{PC7} & RF chip enable &  & Chip Enable Activates RX (used as receiver) or TX Mode (used as transmitter) \\
  \hline
  \textbf{PD2} & RF IRQ & Dev Port RX & Maskable interrupt pin. Active low. Confirms either packet arrival when package loaded into RX FIFO (RX mode) or arrival of ACK (acknowledge) packet (TX mode). Additionally this lane is the communication Input of UART Serial Communication (Output atmega32u4 - Input Dev Port) \\
  \hline
  \textbf{PD3} & Dev Port TX &  & Communication Output of UART Serial Communication (Output Dev Port - Input atmega32u4) \\
  \hline
\end{tabular}
\normalsize

\newpage
\section*{Interrupts}

\textit{Which conditions should an interrupt concerning the communication with the expansion board occur?} \\
\textit{Is the interrupt triggered by an external or internal source?} \\

3pi/Scout(ATmega 328): \\
\begin{itemize}
\item Finding a location where the photo sensor readings surpass 200 (external)
\item Check for timer $> 5$ minutes (internal)
\end{itemize}
Zumo/Collector(ATmega 32u4): \\
\begin{itemize}
\item Check if communication by Scout $\neq -1$ (external)
\end{itemize} 

\flushleft
\textit{How does the interrupt service routine handle the interrupt? Which pins are affected?} \\

3pi/Scout(ATmega 328): \\
\begin{itemize}
\item Finding a location: INT0, 0x0002
The interrupt is executed asynchronously 
\item Check for timer: TIMER2 COMPA, 0x000E 
\end{itemize}
Zumo/Collector(ATmega 32u4): \\
\begin{itemize}
\item Check Scout communication: INT0, \$0006
\end{itemize}

\end{document}
