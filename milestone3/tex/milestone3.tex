% exercise sheet with header on every page for math or close subjects
\documentclass[12pt]{article}
\usepackage[utf8]{inputenc}
\usepackage{latexsym}
\usepackage{multicol}
\usepackage{fancyhdr}
\usepackage{amsfonts}
\usepackage{amsmath}
\usepackage{amssymb}
\usepackage{enumerate}
\usepackage{listings}
\usepackage{graphicx}

% Shortcuts for bb, frak and cal letters
\newcommand{\E}{\mathbb{E}}
\newcommand{\V}{\mathbb{V}}
\renewcommand{\P}{\mathbb{P}}
\newcommand{\N}{\mathbb{N}}
\newcommand{\R}{\mathbb{R}}
\newcommand{\C}{\mathbb{C}}
\newcommand{\Z}{\mathbb{Z}}
\newcommand{\Pfrak}{\mathfrak{P}}
\newcommand{\Pfrac}{\mathfrak{P}}
\newcommand{\Bfrac}{\mathfrak{P}}
\newcommand{\Bfrak}{\mathfrak{B}}
\newcommand{\Fcal}{\mathcal{F}}
\newcommand{\Ycal}{\mathcal{Y}}
\newcommand{\Bcal}{\mathcal{B}}
\newcommand{\Acal}{\mathcal{A}}

% formating
\topmargin -3.5cm
\textheight 22cm
\textwidth 16.0 cm
\oddsidemargin -0.1cm

% Fancy Header on every Page
\pagestyle{fancy}
\lhead{\textbf{Embedded Systems Milestone 3}\\Hardware Number 1}
\rhead{Daniel Schäfer (2549458)\\ Rafael Dewes (2548365)\\ Kevin M\"uller (2550062)}
\renewcommand{\headrulewidth}{1.2pt}
\setlength{\headheight}{110pt}

\begin{document}
\pagenumbering{gobble}
\lstset{language=C++}

\section*{Draft: Notes which Tasks exist for both robots}

\subsection*{NOTES: List of Tasks Scout}
\begin{itemize}
  \item Receive photosensor readings (periodic), might queue the task to handle high readings
  \item Handle high photosensor readings (high priority), puts collector communication into task queue
  \item communicate harvesting position to collector (high priority)
  \item Correct Theta (by changing the current velocities of the two motors) to point towards current destination (preemptive, lowest priority, if condition for turning not true - just continue to drive straight). This tasks queries the "generate new destination" task upon reaching the current destination
  \item Generate new Goal (medium priority), generate new destination where to drive - mainly tactic finetuning (initially will generate random value)
  \item Send collector the currently estimated scout position (low priority, periodic)
  \item Receive Referee updates and update estimation (medium priority, triggered by interrupt)
\end{itemize}

\subsection*{Collector Tasks}
\begin{enumerate}
  \item Receive proximity sensor readings and queue task to handle readings above the threshold.
  	\begin{itemize}
  	\item \textbf{Main properties:} Periodic, non-preemtive and not entirely known before-hand
  	\item \textbf{Release time:} Same as the period (immediately available)
  	\item \textbf{Period:} Since this task can be handled very fast, the period can also be very fast. We will start with a period of 10ms and later fine-tune it using the hardware.
  	\item \textbf{Relative deadline:} Does not need a deadline
  	\item \textbf{Approximate cost:} The following operations need to be executed where the last one needs to be done only when the readings are higher than the threshold, which should happen rather rarely: Read each the sensor (3) and for each one compare the value to the threshold (3), enqueue task (i.e. make call to the scheduler) to handle high readings (2). We argue that the first eight tasks (read and compare) can be executed almost immediately and the last one also very fast. Therefore we assign this task a (relative) cost of 3+3+2=8.
  	\item \textbf{Urgency:} The worst thing that happens when the task is delayed for too long it that he robot might not see another robot nearby. This does not do any damage. However, it might also collide with its team mate which would do significant damage so this task has a (relative) urgency of 5.
  	\item \textbf{Dependencies:} No dependencies on other tasks.
  	\end{itemize}
  \item Handle high proximity readings, includes calculation if its own scout.
  	\begin{itemize}
  	\item \textbf{Main properties:} Aperiodic, non-preemtive, not entirely known before-hand
  	\item \textbf{Release time:} Becomes available as soon as enqueued by other task. (= 0 or not applicable?)
  	\item \textbf{Relative deadline:} Depending on the threshold, this task should get a deadline that ensures that the collector will never push its team mate. Since we have not fixed the threshold proximity and maximum driving speed, the exact deadline can only be calculated later when working with the real hardware.
  	\item \textbf{Approximate cost:} The following operations need to be executed: Retrieve last scout position (1), retrieve time and calculate elapsed time since last scout position update (3), calculate the distance that he scout has traveled since (i.e. the radius where scout might be) (2), compare current proximity value with possible scout position (3) and make decision whether to hit or not and depending on the result set the current destination value (1). Therefore we assign this task a (relative) cost of 1+3+2+3+1=10. 
  	\item \textbf{Urgency:} This task has about he same reasoning for urgency as Task 1, so it also gets an urgency of 5.
  	\item \textbf{Dependencies:} This task can only be executed after Task 1. This is ensured because this task is scheduled at the end of Task 1 when all rw operations to the shared global memory are done and the task is finished.
   	\end{itemize} 	
  \item Correct Theta (by changing the current velocities of the two motors) to point towards current destination (preemptive, lowest priority, if condition for turning not true - just continue to drive straight). This tasks queries the "generate new destination" task upon reaching the current destination
  \begin{itemize}
  	\item \textbf{Main properties:} Periodic, preemtive, not entirely known before-hand
  	\item \textbf{Release time:} Same as the period (immediately available)
  	\item \textbf{Relative deadline:} Does not need a deadline
  	\item \textbf{Approximate cost:} The cost is proportional to the radius that the robot needs to turn and depends on the wheel velocities we choose. In any case, this task takes a lot longer than all other tasks because a physical motion and calculation of the new rotation needs to be done. We could set the cost of this task to be 20 per degree the robot needs to turn.
  	\item \textbf{Urgency:} If the task gets delayed for too long we might miss the target destination and in the worst case cross the arena border. However, for this to happen the task would have to be delayed for a very long time which is why it will get a relatively low urgency of 1.
  	\item \textbf{Dependencies:} The robot needs a destination before executing this task. Therefore, Task 5 has to finish before this task starts for the first time. This task queues Task 4 upon reaching the current destination.
   	\end{itemize} 	
  \item Generate new Goal (medium priority), generate new destination where to drive - mainly tactic finetuning (initially will generate random value)
  \begin{itemize}
  	\item \textbf{Main properties:} Aperiodic, preemtive, not entirely known before-hand.
  	\item \textbf{Release time:} Becomes available as soon as enqueued by other task. (= 0 or not applicable?)
  	\item \textbf{Relative deadline:} Does not need a deadline.
  	\item \textbf{Approximate cost:} The cost highly depends on the tactic implemented. Since this can possibly take a long time, the task needs to be preemtive. For not, we will just randomly generate valid destinations. A simple implementation using the clock (1) and some arithmetic operations (5) could generate such destinations with a cost of 2 x (1 + 5) = 12.
  	\item \textbf{Urgency:} It is never a problem when the task gets delayed for some time because the robot will never drive blindly without having a destination. So, when no new destination gets generated for some time the robot will simply remain at its position. The only loss would be that we can not execute our tactic but that is not a big problem compared to hitting team mates or crossing the arena border. Therefore, just as Task 4 this task gets an urgency of 1.
  	\item \textbf{Dependencies:} This task is executed when Task 4 is finished and the current destination is reached.
   	\end{itemize}
  \item Receive Referee updates and update estimation.
    \begin{itemize}
  	\item \textbf{Main properties:} Aperiodic (triggered by interrupt), non-preemtive (low cost), entirely known before-hand.
  	\item \textbf{Release time:} Immediately available after interrupt.
  	\item \textbf{Relative deadline:} Does not need a deadline.
  	\item \textbf{Approximate cost:} The robot needs to read the data from the RF module (1) and extract both x and y positions as well as the angle (3). It then update its internal estimated position to match the real position (3). It therefore gets a cost of 7.
  	\item \textbf{Urgency:} Driving with an incorrent estimation of the real position is very bad. Considering that collisions occur very often, it is extremly important to update the estimation as soon as possible. It therefore gets a relatively high urgency of 10.
  	\item \textbf{Dependencies:} No dependencies.
   	\end{itemize}
  \item Handle received harvesting position (highest priority, triggered by interrupt). Overwrites current destination regardless of roboter state.
  \begin{itemize}
  	\item \textbf{Main properties:} Aperiodic (triggered by interrupt), non-preemtive (low cost), not entirely known before-hand.
  	\item \textbf{Release time:} Immediately available after interrupt.
  	\item \textbf{Relative deadline:} Does not need a deadline.
  	\item \textbf{Approximate cost:} The robot needs to read the data from the RF module (1) and extract both x and y positions of the harvest position (2). It then updates the current destination (2). Therefore, cost = 4.
  	\item \textbf{Urgency:} Since the light moves very fast, it is important to react very fast to new harvest positions. When the collector waits for too long, the light might have moved somewhere else. It therefore gets a relatively high urgency of 10.
  	\item \textbf{Dependencies:} No dependencies.
   	\end{itemize}
\end{enumerate}


\end{document}
