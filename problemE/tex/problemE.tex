% exercise sheet with header on every page for math or close subjects
\documentclass[12pt]{article}
\usepackage[utf8]{inputenc}
\usepackage{latexsym}
\usepackage{multicol}
\usepackage{fancyhdr}
\usepackage{amsfonts}
\usepackage{amsmath}
\usepackage{amssymb}
\usepackage{enumerate}
\usepackage{listings}
\usepackage{graphicx}
\usepackage[parfill]{parskip}

% Shortcuts for bb, frak and cal letters
\newcommand{\E}{\mathbb{E}}
\newcommand{\V}{\mathbb{V}}
\renewcommand{\P}{\mathbb{P}}
\newcommand{\N}{\mathbb{N}}
\newcommand{\R}{\mathbb{R}}
\newcommand{\C}{\mathbb{C}}
\newcommand{\Z}{\mathbb{Z}}
\newcommand{\Pfrak}{\mathfrak{P}}
\newcommand{\Pfrac}{\mathfrak{P}}
\newcommand{\Bfrac}{\mathfrak{P}}
\newcommand{\Bfrak}{\mathfrak{B}}
\newcommand{\Fcal}{\mathcal{F}}
\newcommand{\Ycal}{\mathcal{Y}}
\newcommand{\Bcal}{\mathcal{B}}
\newcommand{\Acal}{\mathcal{A}}

% formating
\topmargin -3.5cm
\textheight 22cm
\textwidth 16.0 cm
\oddsidemargin -0.1cm

% Fancy Header on every Page
\pagestyle{fancy}
\lhead{\textbf{Embedded Systems Problem Set E}}
\rhead{Daniel Schäfer (2549458) \\ Kevin M\"uller (2550062) \\ Rafael Dewes (2548365) }
\renewcommand{\headrulewidth}{1.2pt}
\setlength{\headheight}{110pt}

\begin{document}
\pagenumbering{gobble}
\lstset{language=C++}

\section*{Problem E1}
\subsection*{a)}
(7, 8, 9, 1, 2, 3, 4, 5, 3, 4, 6)

\subsection*{b)}
(2, 7, 8)

\subsection*{c)}
$\{$(1, 2, 7, 9, 1), (7, 8, 9, 1, 2, 3, 4, 6), (3, 4, 5, 3)$\}$

\subsection*{d)}
For path coverage of all finite paths the infinite set of paths that fulfill the following suffice:
$\{ p \in $ (1, 2, 7, (8, $|$ 8, 9,))* 2, (3, 4, 5,)* 3, 4, 6 $\}$ ; where * denotes any finite amount of repetitions

Since we cannot cover all infinite paths with our test paths, we do not have full path coverage.

This infinte set covers all finite paths, therefore we do achieve prime path coverage. 
\section*{Problem E2}
%TODO problem e2


\section*{Problem E3}
\subsection*{a)}

%TODO Automaton

\subsection*{b)}
\begin{itemize}
	\item[i)] $ \mathbf{G}( \neg \: bdep )$
	\item[ii)] $ \mathbf{G} ( \neg home \rightarrow \mathbf{F} \: home ) $
	\item[iii)] $ \mathbf{G} (pu2 \rightarrow ( \neg pu1 \: \mathbf{U} \: compacting ) ) $
	\item[iv)] $ \mathbf{G} \neg (fis \rightarrow \neg \mathbf{F} \: puf )$
\end{itemize}

\subsection*{c)}
\begin{itemize}
	\item[i)]
	\item[ii)]
	\item[iii)]
	\item[iv)]
\end{itemize}


\end{document}
