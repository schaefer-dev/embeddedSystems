% exercise sheet with header on every page for math or close subjects
\documentclass[12pt]{article}
\usepackage[utf8]{inputenc}
\usepackage{latexsym}
\usepackage{multicol}
\usepackage{fancyhdr}
\usepackage{amsfonts}
\usepackage{amsmath}
\usepackage{amssymb}
\usepackage{enumerate}
\usepackage{listings}
\usepackage{graphicx}

% Shortcuts for bb, frak and cal letters
\newcommand{\E}{\mathbb{E}}
\newcommand{\V}{\mathbb{V}}
\renewcommand{\P}{\mathbb{P}}
\newcommand{\N}{\mathbb{N}}
\newcommand{\R}{\mathbb{R}}
\newcommand{\C}{\mathbb{C}}
\newcommand{\Z}{\mathbb{Z}}
\newcommand{\Pfrak}{\mathfrak{P}}
\newcommand{\Pfrac}{\mathfrak{P}}
\newcommand{\Bfrac}{\mathfrak{P}}
\newcommand{\Bfrak}{\mathfrak{B}}
\newcommand{\Fcal}{\mathcal{F}}
\newcommand{\Ycal}{\mathcal{Y}}
\newcommand{\Bcal}{\mathcal{B}}
\newcommand{\Acal}{\mathcal{A}}

% formating
\topmargin -3.5cm
\textheight 22cm
\textwidth 16.0 cm
\oddsidemargin -0.1cm

% Fancy Header on every Page
\pagestyle{fancy}
\lhead{\textbf{Embedded Systems Problem Set B}}
\rhead{Daniel Schäfer (2549458)\\ Rafael Dewes (2548365)\\ Kevin M\"uller (2550062)}
\renewcommand{\headrulewidth}{1.2pt}
\setlength{\headheight}{110pt}

\begin{document}
\pagenumbering{gobble}
\lstset{language=C++}


\section*{Problem B1}
% TODO
\begin{enumerate}[a)]
	\item According to sampling theory (Nyquist criterion) the sampling rate needs to be at least twice as big as the highest frequency of the component sine waves.
	$$ f_s > 2 * f_N$$
	To compute the necessary sampling rate, the Nyquist frequency $f_N$ has to be computed as follows
	$$ S(t) = \underbrace{sin(2\pi t + \frac{1}{4}}_{f = 1Hz} + \underbrace{sin(4\pi t + \frac{1}{4}}_{f = 2}$$
	The highest frequency of the component sine waves is $2$, which means that $f_N = 2$.
	
	$\Rightarrow$ The signal has to be sampled at least 4 times per second to enable a reconstruction (Nyquist frequency).

\end{enumerate}
\section*{Problem B2}
% TODO


\section*{Problem B3}
% TODO


\section*{Problem B4}
\begin{enumerate}[a)]
	\item The integral part of a PID (or PI) controller accumulates the error so far, meaning that the constant error introduced by the bias will accumulate. Eventually, by adjusting the integral gain as a response to the accumulation of errors, the error will go to zero. The derivative part of a PID (or PD) controller changes the signal gain based on the error rate of change. Since we have a constant bias, the rate of change will essentially be zero. Therefore a PD controller will have no advantage over simple P controller (only proportional gain), while a PI controller will get rid of the bias and get the error down to zero.\\
This is our stateflow diagram of the differential drive:\\
Here is our simulink block diagram of the plant and controller:\\

	\item TODO
\end{enumerate}

\end{document}
